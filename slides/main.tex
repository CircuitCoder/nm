\documentclass{ctexbeamer}
\usepackage{changepage}
\usetheme{Berkeley}
\title{nm}
\date{2022年12月19日}

\begin{document}

\frame{\titlepage}

\section{背景}
\begin{frame}
\frametitle{网络配置}
\begin{block}{网络配置包含什么}
\begin{itemize}
\item 接口
\item 地址
\item 路由
\item DNS
\end{itemize}
\end{block}
\begin{exampleblock}{配置方式}
\begin{itemize}
\item 写文件(例如\texttt{/etc/resolv.conf})
\item iproute2
\item NetworkManager, systemd-networkd, ...
\end{itemize}
\end{exampleblock}
\end{frame}

\begin{frame}[fragile]
\frametitle{iproute2}
\begin{verbatim}
Usage: ip [ OPTIONS ] OBJECT { COMMAND | help }
where  OBJECT := { address | addrlabel | amt | fou
                 | help | ila | ioam | l2tp | link
                 | macsec | maddress | monitor
                 | mptcp | mroute | mrule
                 | neighbor | neighbour | netconf
                 | netns | nexthop | ntable | ntbl
                 | route | rule | sr | tap
                 | tcpmetrics | token | tunnel
                 | tuntap | vrf | xfrm }
\end{verbatim}
\begin{alertblock}{netlink}
iproute2可以管理的资源很多,但其本质通过的是内核提供的netlink接口,与修改文件等配置有本质的不同
\end{alertblock}
\end{frame}

\begin{frame}[fragile]
\frametitle{netlink}
\begin{quotation}
  Netlink is used to transfer information between the kernel and
  user-space processes. It consists of a standard sockets-based
  interface for user space processes and an internal kernel API for
  kernel modules.

  netlink\_family selects the kernel module or netlink group to
  communicate with. The currently assigned netlink families are:

  \begin{adjustwidth}{0.5cm}{}
    NETLINK\_ROUTE
  \end{adjustwidth}

  \begin{adjustwidth}{1cm}{}
    Receives routing and link updates and may be used to
    modify the routing tables (both IPv4 and IPv6), IP
    addresses, link parameters, neighbor setups, queueing
    disciplines, traffic classes, and packet classifiers (see
    rtnetlink(7)).
  \end{adjustwidth}

  \ldots
\end{quotation}
\end{frame}

\end{document}
